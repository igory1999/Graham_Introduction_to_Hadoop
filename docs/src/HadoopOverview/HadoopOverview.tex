\documentclass{beamer}

\usepackage[utf8]{inputenc}
\usepackage{default}

\mode<presentation>
%{ \usetheme{boxes} }


\usetheme{Madrid}

\usepackage{times}
\usepackage{graphicx}
\usepackage{tabulary}
\usepackage{listings}
\usepackage{verbatimbox}
\usepackage{graphicx}
\usepackage{lmodern}
\usepackage[absolute,overlay]{textpos}
\usepackage{pgfpages}
\usepackage{color}

\pgfdeclareimage[height=1.0cm]{logo_rcc}{../icons/logo_rcc.png}
\setlength{\TPHorizModule}{1mm}
\setlength{\TPVertModule}{1mm}
\newcommand{\RCCLogo}{
\begin{textblock}{14}(1.5,1.5)
  \pgfuseimage{logo_rcc}
\end{textblock}
}

\definecolor{mycolorcli}{RGB}{53,154,26}
\definecolor{mycolorcode}{RGB}{0,0,255}
\definecolor{mycolordef}{RGB}{255,0,0}
\definecolor{mycolorlink}{RGB}{184,4,255}

\title{\huge{Overview of Hadoop}}
\author{Igor Yakushin \\ \texttt{ivy2@uchicago.edu}}
%\date{January 20, 2018}

\definecolor{ChicagoMaroon}{RGB}{128,0,0}

\setbeamercolor{title}{bg=ChicagoMaroon}

\begin{document}

\setbeamertemplate{navigation symbols}{}

\setbeamercolor{fcolor}{fg=white,bg=ChicagoMaroon}
\setbeamertemplate{footline}{
\begin{beamercolorbox}[ht=4ex,leftskip=1.4cm,rightskip=.3cm]{fcolor}
\hrule
\vspace{0.1cm}
   \hfill \insertshortdate \hfill \insertframenumber/\inserttotalframenumber
\end{beamercolorbox}
}

\setbeamercolor{frametitle}{bg=ChicagoMaroon,fg=white}

\begin{frame}
\titlepage
\end{frame}


\section{What is Hadoop?}
\begin{frame}
  \frametitle{What is Hadoop?}
  \begin{itemize}
   \item A framework to process huge amount of data
   \item Implemented in Java
   \item Runs on a cluster of commodity computers, from a few to a few thousand nodes
   \item Consists of two major components:
    \begin{itemize}
      \item {\color{mycolordef}HDFS} - distributed file system; everything you put there is automatically divided into blocks that are replicated and spread across the cluster;
      \item {\color{mycolordef}MapReduce} - an approach to perform calculations on such data in parallel.
    \end{itemize}
  \end{itemize}
\end{frame}

\begin{frame}
  \frametitle{Hadoop zoo}
  \begin{itemize}
  \item There is a zoo of other Hadoop related tools
    \begin{itemize}
    \item {\color{mycolordef}Hive} - SQL interface to Hadoop built on top of HDFS and MapReduce
    \item {\color{mycolordef}Impala} - another SQL interface to Hadoop, allegedly much faster
    \item {\color{mycolordef}HBase} - noSQL fast distributed database on top of HDFS
    \item {\color{mycolordef}Pig}  -  data processing languague built on top of HDFS and MapReduce
    \item {\color{mycolordef}Spark} - for performance reasons no longer relies on MapReduce,
      can be used without Hadoop; its native language is Scala but it can also be used from Java, Python, R
    \item {\color{mycolordef}Sqoop} - copy data between relational database and Hadoop
    \item {\color{mycolordef}Flume} - copy data into Hadoop, typically used to store logs from a cluster for subsequent analysis
    \item {\color{mycolordef}YARN} - resource allocation manager, used to submit jobs
    \item {\color{mycolordef}Zookeeper} - centralized service for maintaining configuration information, naming, providing distributed synchronization
    \item {\color{mycolordef}Hue} - web GUI to many of the above tools
    \end{itemize}
  \item In Big Data Platform class you'll mostly use pySpark via JupyterHub interface, HDFS, Hive, Pig, Hue.
  \end{itemize}
\end{frame}

\section{Hadoop vs traditional RDBMS}
\subsection{Comparison}
\begin{frame}
  \frametitle{Hadoop vs traditional RDBMS}
  \begin{center}

    \begin{tabulary}{\textwidth}{L | L}
      \hline
      {\color{mycolordef}\textbf{RDBMS}} & {\color{mycolordef}\textbf{Hadoop}} \\ \hline
      do not scale well beyond a few terabytes & easily scales to petabytes by adding more computers to the cluster \\ \hline
      for large datasets runs on very expensive enterprise server & data is scattered on a cluster of cheap commodity computers \\ \hline
      works on structured data; one needs to predefine the data schema & can accept any unstructured data and worry about interpreting and reinterpreting it later \\ \hline
      data needs to be normalized to avoid duplication and enforce constraints & works best on a single denormalized table \\ \hline

    \end{tabulary}

  \end{center}
\end{frame}

\begin{frame}
  \frametitle{Hadoop vs traditional RDBMS}
  \begin{center}

    \begin{tabulary}{\textwidth}{L | L}
      \hline
      {\color{mycolordef}\textbf{RDBMS}} & {\color{mycolordef}\textbf{Hadoop}} \\ \hline
      can be faster for queries on small subset of data & might introduce too much overhead for such queries \\ \hline
      ACID - compliant & in general - not ACID-compliant \\ \hline
      natively speaks SQL & natively implements MapReduce approach in Java on top of which some subset of SQL might be supported \\ \hline 
      is good for banks to keep track of transactions & is good for data scientists to try various ideas on a huge sets of data  \\ \hline
    \end{tabulary}

  \end{center}
\end{frame}

\subsection{ACID compliance}
\begin{frame}
`\frametitle{ACID compliance}
\begin{itemize}
  \item {\color{mycolordef}\textbf{A}}tomicity - The database transaction must completely succeed or completely fail
  \item {\color{mycolordef}\textbf{C}}onsistency - During the database transaction, RDBMS progresses from one valid state to another. The state is never invalid
  \item {\color{mycolordef}\textbf{I}}solation - The client's database transaction must occur in isolation from other clients attempting to transact
  \item {\color{mycolordef}\textbf{D}}urability - Once transaction is committed, it will remain so, 
    even in the event of power loss, crashes, or errors. The data operation that was part of the transaction must be reflected in nonvolatile storage. 
\end{itemize}

Hadoop has no concept of transaction so is not ACID compiliant.

\end{frame}

\section{Distributions}
\begin{frame}
 \frametitle{Distributions}
 There are many Hadoop distributions that bundle different set of tools and add their own:
 \begin{itemize}
  \item {\color{mycolordef}Apache}  - free, open source;
  \item {\color{mycolordef}Cloudera} - we are using Cloudera 6.3;
  \item {\color{mycolordef}Hortonworks} - Hortonworks and Cloudera recently merged but they still maintain two separate distributions and working on the joint one
  \item {\color{mycolordef}HDInsight} - Microsoft  Azure's Cloud based Hadoop Distribution
  \item {\color{mycolordef}MapR}
  \item ...
  \end{itemize}
  Many distributions, for example Cloudera, provide virtual machines to play with
\end{frame}


\end{document}
